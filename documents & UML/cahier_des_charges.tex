\documentclass{article}


\usepackage[french]{babel}
\usepackage[utf8]{inputenc}
\usepackage{graphicx}

\title{Todo list : cahier des charges}
\date{\today}
\author{Lucas Franceschino}

\begin{document}
	\maketitle

	\tableofcontents

	\section{Introduction}
	Ce programme est un gestionnaire de tâches écris en Java, dans le cadre de l'UE HLIN505, à la fac de science de Montpellier.\\

	\section{Analyse des besoins et choix}
		\subsection{Généralités}
			On shouaite développer un gestionaire de tâche à usage individuel. Lorque quelqu'un souhaite organiser, lister et gérer ses tâches, c'est qu'il souhaite optimiser son temps et mieux gérer les imprévus.
		\subsection{Tâches}
			Le concept majeur de ce programme est le concept de tâche.
			\begin{itemize}
				\item Une tâche possède un nom non vide ;
				\item Une tâche est rangée dans une ou aucune catégorie ;
				\item Une tâche peut avoir une description ;
				\item A tout instant, tant que la tâche n'est pas terminée, on peut modifier ces trois précédentes caractéristiques ;
				\item Une tâchge a un moment d'échéance ;
				\item Ce moment est d'une précision d'un jour : la granularité des date est de l'ordre du jour, donc ;
				\item Il n'est pas possible de créer une tâche déjà en retard$_{(1)}$ (la date d'échéance d'une tâche ne peut être que strictement ultérieur au jour actuel) ;
				\item Une tâche peut se trouver dans deux états : terminé ou non ;
				\item Une tâche peut se trouver dans deux états : en retard ou non ;
				\item Une tâche doit pouvoir être supprimée ;
				\item Une tâche peut être de deux types différents :
				\begin{itemize}
					\item Court : possède une date d'échéance simple ;
					\item Long : possède une date de début et une date de fin.\\
						La date de début doit nécéssairement être avant la date de fin$_{(2)}$.\\
						On considère toujours 3 paliers entre le début et la fin :
						\begin{itemize}
							\item \textit{Tâche finie à 0\% (Date de fin)}
							\item Tâche finie à 25\%
							\item Tâche finie à 50\%
							\item Tâche finie à 75\%
							\item \textit{Tâche finie à 100\% (Date de fin)}
						\end{itemize}
						Pour ce type de tâche, on considère un pourcentage entier représentant l'avancement du projet.\\La granularité de l'avancement est ici 1 pourcent.\\
						On ne peut que avancer dans l'avancement. 
				\end{itemize}
				\item On peut modifier ces tâches, tant que les conditions $_{(1,2)}$ sont respectées ; 
			\end{itemize}
			\subsubsection{Retard d'une tâche}
			Une tâche est signaliée comme en retard si sa date d'échéance, ou si elle en possède, sa date d'échéance intermédiaire, est dépassée.
			\subsubsection{Tâche terminée}
			Une tâche accomplie signifie :
			\begin{itemize}
				\item Un pourcentage de 100\% d'accomplissement dans le cas d'une tâche longue ;
				\item Dans le cas d'une tâche courte, une date d'accomplissement effective d'une tâche non nulle, affecté lorsque l'utilisateur le décide.
			\end{itemize}
		\subsection{Catégories}
			Pour gérer correctement des tâches, il est intéressant d'en faire une découpe disjointe, de façon à les ranger dans différentes catégories.
			\\~\\
			Une tâche (longue ou courte) peut :
			\begin{itemize}
				\item appartenir à une catégorie
				\item ne pas appartenir à une catégorie, et dans ce cas, être rangé parmis les tâche non catégorisés, ce qui peut être vu comme une autre catégorie.
			\end{itemize}~\\
			Une catégorie doit suivre le comportement suivant :
			\begin{itemize}
				\item Son titre ne doit pas être nul ;
				\item Si on modifie son titre, alors toute partie de l'interface présentant cette catégorie doit être immédiatement rafraichie pour correspondre à la nouvelle version ;
				\item Si on supprime une catégorie, toute tâches sous cette précédente catégorie doit être désafectée de cette tâche et rangé sous "non catégorisés".
			\end{itemize}
		\subsection{Interface}
			L'interface devra présenter en premier plan une liste de tâches à effectuer.
			\paragraph{}
			En deuxième plan, on souhaire accéder aux fonctions suivantes, par le biais de boutons directement accessible :
			\begin{itemize}
				\item la liste éditable des catégories ;
				\item ajouter une catégorie ;
				\item éditer un bilan.
			\end{itemize}
			\paragraph{}
			Cette liste de tâches à effectuer pourra apparaitre sous trois formes différentes :

			\begin{itemize}
				\item \textbf{Simple} : les tâches sont affichées par ordre croissant de date d'échéance. Visuellement, les tâches partageant la même date d'échéance sont regroupées.
				\item \textbf{Priorité} : les tâches sont affichées par ordre croissant de date d'échéance, en considérant les paliers des dates à long termes comme des dates d'échéances. Visuellement, les tâches partageant la même date d'échéance sont regroupées.
				\item \textbf{Résumé} : affiche une tâche importante, trois tâches moyennement importantes, et 5 peu importante. Visuellement, les tâches partageant un même degré de priorité sont regroupées.
			\end{itemize}
			\paragraph{}
			Un code couleur sera utilisé pour spécifier qu'une tâche est en retard.
			\paragraph{}
			Un menu contextuel sur la tâche permet de la supprimer.
			\paragraph{}
			Pour les tâches de long court, on utilisera une représentation graphique pour montrer où l'on en est.
		\subsubsection{Catégories}
		Dans l'interface, les catégories seront présentées en liste, avec possibilité d'édition, de rajout, et de suppression. Cette liste sera présenté par le biais d'un menu coulissant latéral à gauche.
		\paragraph{}
		Depuis ce menu, il sera aussi possible de choisir un filtre en terme de catégorie pour l'affichage des tâches.
	\subsection{Edition de bilans}
		On souhaite éditer des bilans.
		\paragraph{(dépendance)} Pour pouvoir éditer des bilans, il faut pouvoir accéder à la liste des tâches qui ont déjà étés complétées, c'est-à dire qu'il ne faut pas supprimer mais conserver les tâches complétées.
		\paragraph{}
		Soit $d_1$ et $d_2$, deux dates, avec $d_1<d_2$. Un bilan pour ces dates est un résumé de toutes les tâches dont l'échéance $d_e\in [d_1, d_2]$.
		\paragraph{}
		Un bilan propose le nombre de tâche :
		\begin{itemize}
			\item Terminée en retard ;
			\item Terminée à l'heure ;
			\item Pas encore terminée mais en retard ;
			\item Pas encore terminée mais pas en retard ;
		\end{itemize}
		\paragraph{}
		Aussi le bilan montre ces différentes tâches.
		\paragraph{}
		Un bilan est intéressant s'il est exportable, j'ai donc choisi le format HTML, facile à générer et très portable, pour présenter les fonctionnalités précédement caractérisées. 
	\subsection{Données}
		Les données devront pouvoir être sauvegardées, être presistantes.
\section{Choix graphiques}
	Je me suis inspiré de Todoist, gestionnaire de tâche que j'adore, \url{https://fr.todoist.com/}.
\end{document}